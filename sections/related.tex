\section{Related Work}

\textbf{Learning from Wikipedia Edit History.}
Wikipedia edit history has been exploited in a number
of language understanding problems.
Prior methods target various tasks different from ours.
  A popular task in this regard is that of
Wikipedia edit history categorization\cite{daxenberger2013automatically}.  This task
involves characterizing  a given edit instance as one of many possible categories 
such as spelling error correction, paraphrase or vandalism. 
\cite{DaxenbergerG12}  came up with a 21 category edit classification
taxonomy.  Other tasks to leverage Wikipedia edit history include: sentence compression, bias detection, and
 textual  entailment \cite{Nelken08miningwikipedia,Cahill13robustsystems,Zanzotto_expandingtextual,RecasensDJ13}.
These studies are concerned with coarse grained change type classification as opposed
to  establishing a verb-level  correspondence between text changes and  infobox changes.

\textbf{Learning State Changing Verbs.}
Very few works have studied the problem of learning state changing verbs.
\cite{HosseiniHEK14} learned state changing verbs in the context of solving arithmetic word problems.
They learned the effect of the words such as add, subtract on  the current state. 
The VerbOcean resource was automatically generated from the Web\cite{Chklovski04}. The authors  studied the problem of fine-grained semantic relationships between verbs. They learn relations such as  if someone has bought an item, they may sell it at a later time. This then involves capturing empirical regularities such as  ``X buys Y'' happens before ``X
sells Y''. Unlike the work we present here, the methods of \cite{Chklovski04,HosseiniHEK14}  do not make a connection to knowledge base relations such as Wikipedia infoboxes.
In a vision paper, \cite{Wijaya2014akbc} give high level descriptions of  a number of possible methods for learning state changing methods. They  did not implement any of them.
