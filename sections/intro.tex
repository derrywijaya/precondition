\section{Introduction}

\begin{figure*}[t]
\begin{center}
\includegraphics[width=16cm,keepaspectratio=true]{figures/motivation.pdf}
\caption{\label{fig:motivation} A snapshot of Kim Kardashian's Wikipedia revision history, highlighting text and infobox changes. In red (and green) are the difference between the page on 05/25/2014 and 05/23/2014: things that are deleted from (and resp. added to) the page.}
\end{center}
\end{figure*}

In recent years there has been a lot of research on extracting relational facts between entities and storing them in knowledge bases (KBs) . These knowledge bases are generally static  \cite{suchanek2007yago,carlson2010toward,fader2011identifying,MitchellCHTBCMG15}
%(which extract facts from Wikipedia infoboxes \cite{suchanek2007yago}) or NELL (which extracts facts from any Web text \cite{carlson2010toward,fader2011identifying}) are generally static.
 They are not updated as the Web changes.
 % when in reality new facts arise while others cease to be valid%or change over time
One approach towards real-time population of KBs is to extract facts from dynamic content of the web such as news \cite{nakashole2012real}. This paper proposes to lean to identify state changes caused by  verbs acting on entities in text. This is different from simply applying the same text extraction pipeline, that created the original KB, to new datasets.
%a \textit{shift} of focus from doing KB updates by extracting facts in text to doing them by 

The benefit of our approach is as follows: (1) In relation extraction, if we consider for example the \textsc{spouse} relation, both \textit{marry} and \textit{divorce} are good patterns for extracting the relation. Therefore, by identifying that they cause different state changes.
%Ndapa: I think most people know this, so we can skip it to keep the text concise
% \textit{marry} signals the start while \textit{divorce} signals the end of the \textsc{spouse} relation; 
 we can update the entity's fact \textit{and} its temporal scope \cite{wijayactp}. (2) Learning state changing verbs
 %  about by verbs 
   can pave ways to learning the ordering of verbs in terms of their  pre- and post-conditions.
   
   % %Ndapa:
  % I think we are being too speculative here. This was fine in a vision paper such as AKBC but here
  % it might seem like over promising. It is therefore better to not ay much here. 
  
   % of state-changing verbs: the entry condition (in terms of KB facts) that must be true for an event expressed by the verb to take place, and the exit condition (in terms of KB facts) that will be true after the event. Such pre- and post-conditions can be useful for (a) learning event sequences %such as scripts \cite{schank2013scripts}, which can be modeled
%as a collection of verbs chained together by pre- and post-condition of their shared entities, (b) for inferring cascading effect of an event via the pre- and post-condition of shared entities in an event sequence, or (c) for inferring unknown states of entities from the verbs they participate in.  

In this paper, we propose to learn state changing verbs using Wikipedia revision history. Our assumption is that when a state-changing event happens to an entity e.g., a marriage, its Wikipedia infobox is updated, by the addition of a new \textsc{spouse} value. The infobox in Wikipedia is  a structured box on the page that contains a set of facts (attribute-value pairs) about the entity. 

At the same time, texts containing verbs that express the event e.g., \textit{wed} may be added to the entity's Wikipedia page.  Figure \ref{fig:motivation} is an example of this happening to an entity's page. Wikipedia revisions over many entities can act as distantly supervision  data for mapping corresponding text and infobox changes. However, these revisions are notoriously  \textit{noisy}.  Many infobox slots can be  updated when a single event happens.
%there is no guarantee that only the infobox slots related to a particular event will be updated. 
For example, when a death happens, slots regarding birth e.g., \textit{birthdate}, \textit{birthplace}, may also be updated. To address these issues, we leverage common sense constraints between state changes e.g., that the start of \textit{deathdate} is mutually exclusive from the \textit{birthdate} or that the start of \textit{birthdate} is simultaneous with the start of \textit{birthplace}.
%to effectively learn infobox changes that relate to a particular event-expressing verb. 

In summary, our contributions are as follows: (1) the preparation and use of  distantly labeled dataset from Wikipedia revision history for learning state changing verbs, and (2) a resource that contains verbs for identifying state changes, which we make available for future research \footnote{URL retracted for blind reviews}.
% learned resource of verbs that is effective for identifying state changes\footnote[1]{We make our dataset and verbs resource available here: http://.../verbs.html}.