\section{Experiments}

We use 90\% of our labeled documents as train and test on the remaining 10\%. Since our data is noisy, we manually go through our test data to discard documents that have incorrect label to the change in its text. The task is to predict for each document, the label of the document given its verbs features. We compute precision, recall, and F1 values of our predictions of the test set and compare the values before and after feature selection (Table \ref{table:performance}).

\begin{table}
\begin{small}
\begin{center}
\begin{tabular}{|l|r|r|r|}
\hline
Method & Precision & Recall & F1 \\
\hline
\textsc{MaxEnt} & 0.82 & 0.79 & 0.80 \\
\hline
\textsc{MaxEnt} + MIP & 0.88 & 0.76 & 0.82 \\
\hline
\end{tabular}
\caption{\label{table:performance} Results of predicting state change label using verb features.}
\end{center}
\end{small}
\end{table}

We observe the value of doing feature selection by asserting constraints in an MIP formulation in Table \ref{table:performance}. Feature selection improves precision without reducing too much recall; thus resulting in a better F1. Some inconsistent verb features for the labels were removed by asserting constraints. For example, before feature selection, the verbs: ``marry", ``marry in" and ``be married to" were high-weighted features for both \textit{begin-spouse} and \textit{end-spouse}. After asserting constraints that \textit{begin-spouse} is mutex with \textit{end-spouse}, these verbs (whose base form is ``marry") are filtered out from the features of \textit{end-spouse}. We show some of the learned verb features (after feature selection) for some state change labels in (Table \ref{table:verbs}).
\begin{table}
\begin{scriptsize}
\begin{center}
\begin{tabular}{|l|l|}
\hline
Label & Verb \\
\hline
\textit{begin-deathdate} &+(arg1) die on (arg2), +(arg1) die (arg2), +(arg1) pass on (arg2) \\
\hline
\textit{begin-deathplace} &+(arg1) die in (arg2), +(arg1) die at (arg2), +(arg1) move to (arg2) \\
\hline
\textit{begin-birthplace} &+(arg1) be born in (arg2), +(arg1) bear in (arg2), \\
&+(arg1) be born at (arg2) \\
\hline
\textit{begin-} &+(arg1) succeed (arg2), +(arg1) replace (arg2), \\
\textit{predecessor}& +(arg1) join cabinet as (arg2), +(arg1) join as (arg2) \\
\hline
\textit{begin-successor} &+(arg1) lose \textbf{seat} to (arg2), +(arg1) resign on (arg2), \\
& +(arg1) resign from post on (arg2), +(arg1) lose election to (arg2) \\
\hline
\textit{begin-} &+(arg1) work as (arg2), +(arg1) nominate for (arg2), \\
\textit{occupation}& +(arg1) establish as (arg2) \\
\hline
\textit{begin-termstart} &+(arg1) be appointed on (arg2), +(arg1) serve from (arg2), \\
& +(arg1) be elected on (arg2) \\
\hline
\textit{begin-termend} &+(arg1) resign on (arg2), +(arg1) step down in (arg2), \\
& +(arg1) flee in (arg2) \\
\hline
\textit{begin-office} &+(arg1) be appointed as (arg2), \\
& +(arg1) serve as (arg2), +(arg1) be appointed (arg2) \\
\hline
\textit{begin-spouse} &+(arg1) marry on (arg2), +(arg1) marry (arg2), \\
& +(arg1) be married on (arg2), -(arg1) be engaged to (arg2) \\
\hline
\textit{end-spouse} &+(arg1) file \textbf{for divorce} in (arg2), +(arg1) die on (arg2), \\
& +(arg1) divorce in (arg2), +(arg1) announce \textbf{separation} on (arg2) \\
\hline
\textit{begin-children} &+(arg1) have \textbf{child} (arg2), +(arg1) raise daughter (arg2), \\
& +(arg1) raise (arg2) \\
\hline
%\textit{begin-party} &+(arg1) launch (arg2), +(arg1) be elected as (arg2), +(arg1) be elected (arg2) \\
%\hline
\textit{begin-} &+(arg1) graduate from (arg2), +(arg1) attend (arg2), \\
\textit{almamater}& +(arg1) be educated at (arg2) \\
\hline
\textit{begin-awards} &+(arg1) be awarded (arg2), +(arg1) be named on (arg2), \\
& +(arg1) receive (arg2) \\
\hline
\textit{begin-} &+(arg1) start career with (arg2), +(arg1) begin \textbf{career} with (arg2),\\
\textit{youthclubs}& +(arg1) start with (arg2), +(arg1) play for (arg2) \\
\hline
\textit{begin-clubs} &+(arg1) play for (arg2), +(arg1) play during career with (arg2),\\
& +(arg1) sign with (arg2), +(arg1) complete \textbf{move} to (arg2) \\
\hline
\textit{begin-} &+(arg1) make \textbf{appearance} for (arg2), \\
\textit{nationalteam}& +(arg1) make debut for (arg2), +(arg1) play for (arg2) \\
\hline
\end{tabular}
\caption{\label{table:verbs} Comparison of verb features before and after feature selection. The texts in bold are (prep+) noun that occur most frequently with the combination of the (verb, label) in the train data.}
\end{center}
\end{scriptsize}
\end{table}
\normalsize
