\section{Method} \label{sec:method}

\subsection{Data Construction}
We construct a dataset from Wikipedia revision histories of person entities whose facts change between the year 2007 and 2012 (i.e., have at least one fact in YAGO KB with a start or end time in this period). We obtain Wikipedia URLs of this set of entities $P$ from YAGO and crawl their revision histories%, obtaining any revision their pages have between the year 2007 and 2012
. Given a person $p$, his Wikipedia revision history $H_p$ has a set of ordered dates $T_p$ on which revisions are made to his Wikipedia page $W_p$ (we consider a date granularity for time). Each revision $W_{p, t_p} \in H_p$ is the content of $W_p$ at date $t_p$ where $t_p \in T_p$. 

A document $d_{p, t_p}$ in our data set is the \textit{difference}\footnote[2]{a HTML document obtained by ``compare selected revisions"  functionality in Wikipedia} between any two consecutive revisions to $W_p$ that is separated by at least a single date worth of revisions i.e., $d_{p, t_p} = W_{p, \overline{t_p+2}} - W_{p, \underline{t_p}}$. Where $W_{p, \overline{t_p+2}}$ is the \textit{first} revision on date $t_p+2$ and $W_{p, \underline{t_p}}$ is the \textit{last} revision on date $t_p$ (since $W_p$ can be revised multiple times on a date). Our dataset consists of all documents $d_{p, t_p}$, $\forall t_p \in T_p,\ t \in [01/01/2007,\ 12/31/2012]$, and $\forall p \in P$; a total of 288,184 documents from revision histories of 16,909 Wikipedia entities.

%For example, Ralph McInerny's Wikipedia page was consecutively revised on the dates of 11/20/2012, 12/26/2012 and 12/29/2012. We find the difference between the last revision to his page on 11/20/2012 and the first revision to his page on 12/29/2012 (since a page can be revised multiple times in a date). This difference\footnote[2]{http://en.wikipedia.org/w/index.php?title=Ralph\_McInerny \&type=revision\&diff=530257160\&oldid=523980632}, a HTML page obtained by ``compare selected revisions"  functionality in Wikipedia, is a document in our dataset. Using this method, we obtain 288,184 documents from revision histories of 16,909 Wikipedia entities. 

Each Wikipedia revision $W_{p, t_p}$ consists of a set of infobox slots $S$ and a textual content $C$, where each slot $s \in S$ is a quadruple, $\langle s_{att}$, $s_{value}$,  $s_{start}$, $s_{end} \rangle$ containing the attribute name (non-empty), the attribute value, and the start and end time for which this attribute-value pair is valid. 

Each document in our datase is a \textit{difference} between $W_{p, t_p+2} - W_{p, t_p}$, and therefore consists of a set of infobox changes $\Delta S$ and textual changes $\Delta C$. Each slot change $\delta s \in \Delta S$ is also a quadruple %$\langle s_{att}$, $s_{value}$,  $s_{start}$, $s_{end} \rangle$ 
but where $s_{value}$,  $s_{start}$, or $s_{end}$, whenever not empty, is prefixed with $+$ or $-$ to indicate whether they are being added to or deleted in the newer revision $W_{p, t_p+2}$. Similarly, each content change $\delta c \in \Delta C$ is prefixed with $+$ or $-$ indicating whether they are an addition or deletion in $W_{p, t_p+2}$. %For the content, we focus on verbs by extracting (lemmatized) verb phrases from the content change which has subject matched $p$ and object matched $\delta s$ value $s_{value}$ or vice versa.
For example, in Figure \ref{fig:motivation}, a document constructed from the difference $W_{kim, 05/25/2014} - W_{kim, 05/23/2014}$ consists of slot changes: $\langle\textsc{spouse}$, \textbf{$+$}``Kanye West",  $+$``2014", `` "$\rangle$, $\langle\textsc{partner}$, $-$``Kanye West",  $-$``2012-present; engaged", `` "$\rangle$ and content changes: $+$``Kardashian and West were married in May 2014", $-$``She began dating West", $-$``they became engaged in October 2013".

We label documents that have $\langle s_{att}, +s_{value}, *, *\rangle$ or $\langle s_{att}, *, +s_{start}, *\rangle$ $\in \Delta S$ with the label \textit{begin-}$s_{att}$ and documents that have $\langle s_{att}, *, *, +s_{end}\rangle$ $\in \Delta S$ with the label \textit{end-}$s_{att}$. The label represents the state change that happens in the document. Since a document can have more than one slot changes, it can have more than one labels. %if it has more than one $\delta s$ with qualifying value, start or end time. 
For example, in Figure \ref{fig:motivation}, a document constructed from the difference $W_{kim,\tiny 05/25/2014\normalsize } - W_{kim, \tiny 05/23/2014\normalsize }$ is labeled with \textit{begin-spouse} and \textit{end-partner}. 

We use 90\% of our labeled documents as training and test on the remaining 10\%. We focus only on verbs that predict state change, hence for each labeled document we use as features only lemmatized verbs (verb or verb + preposition) in $\delta c$ whose subject matched the person $p$ and whose object matched any slot change value $s_{value}$ in the document (or vice versa). The task is then to predict for a document, the label of the document given its verbs features. 

%We define an infobox attribute of an entity e.g., \textsc{spouse} to \textit{begin} when a new value or a begin time is added to the attribute slot and to \textit{end} when an end time is being added to the slot. Using regular expression to detect whether a new value, a start, or an end time is being added to infobox slots of a document, we automatically label each document with ``begin-\{attribute\_name\}" or ``end-\{attribute\_name\}". So a document that contains an addition of a new value in the \textsc{spouse} slot will be labeled ``begin-spouse", while a document that contains an addition of end time in the \textsc{spouse} slot will be labeled ``end-spouse". 

\subsection{Model}

