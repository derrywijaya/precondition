%
% File emnlp2015.tex
%
% Contact: daniele.pighin@gmail.com
%%
%% Based on the style files for ACL-2015, which were, in turn,
%% Based on the style files for ACL-2014, which were, in turn,
%% Based on the style files for ACL-2013, which were, in turn,
%% Based on the style files for ACL-2012, which were, in turn,
%% based on the style files for ACL-2011, which were, in turn, 
%% based on the style files for ACL-2010, which were, in turn, 
%% based on the style files for ACL-IJCNLP-2009, which were, in turn,
%% based on the style files for EACL-2009 and IJCNLP-2008...

%% Based on the style files for EACL 2006 by 
%%e.agirre@ehu.es or Sergi.Balari@uab.es
%% and that of ACL 08 by Joakim Nivre and Noah Smith

\documentclass[11pt,a4paper]{article}
\renewcommand{\baselinestretch}{1.035}
\usepackage{acl2015}
\usepackage{times}
\usepackage{url}
\usepackage{latexsym}
\usepackage{amsmath}
\usepackage{color}
\usepackage{epsfig,url,algorithm,algorithmic,multirow}
\usepackage{amssymb}

\newtheorem{theorem}{Theorem}
\newtheorem{lemma}{Lemma}
\newtheorem{corollary}{Corollary}
\newtheorem{definition}{Definition}
\newtheorem{example}{Example}
\newtheorem{hypothesis}{Hypothesis}

%\setlength\titlebox{5cm}

% You can expand the titlebox if you need extra space
% to show all the authors. Please do not make the titlebox
% smaller than 5cm (the original size); we will check this
% in the camera-ready version and ask you to change it back.


\title{``A Spousal Relation Begins with a Deletion of \textit{engage} \\and Ends with an Addition of \textit{divorce}": \\
Learning State Changing Verbs from Wikipedia Revision History}
%Ndapa: title is too long, so leavig this out
%\\Mapping Text and Infobox Changes in Wikipedia to \\Learn Verbs and State Changes for Knowledge Base Updates}

\author{%Derry Tanti Wijaya \\
  %Carnegie Mellon University \\
  %5000 Forbes Avenue \\
  %Pittsburgh, PA, 15213 \\
  %{\tt dwijaya@cs.cmu.edu} \\\And
   %Ndapandula Nakashole \\
  %Carnegie Mellon University \\
  %5000 Forbes Avenue \\
  %Pittsburgh, PA, 15213 \\
  %{\tt ndapa@cs.cmu.edu} \\\And
  %Tom M. Mitchell \\
  %Carnegie Mellon University \\
  %5000 Forbes Avenue \\
  %Pittsburgh, PA, 15213 \\
  %{\tt tom.mitchell@cs.cmu.edu} \\
  }

\date{}

\begin{document}

\maketitle

%\vspace{-1cm} %ineffective here anyway

\begin{abstract}
%Ndapa: this abstact intro is a bit stale and over used as we have used it in 
% the other papers, I have changed it a bit and also made the abstract shorter.
% it is better to keep the abstract short and give more motivation in the intro

%Knowledge bases (KBs) %that have emerged 
%that have emerged in recent years  are mostly static. They contain facts about the world yet are seldom updated. This paper proposes a %method for learning state changes brought about by verbs acting on their arguments (i.e., entities).
% State changes are viewed as updates of KB facts pertaining to the entities. 

Learning to determine when the facts of a Knowledge Base (KB) have to be updated is a challenging task.
We propose to learn state changing verbs from Wikipedia revision history. When a state-changing even,  such as a marriage or death,  happens to an entity, the  infobox on the entity's Wikipedia page may be updated. At the same time, the text the same article may be updated with verbs either being added or deleted to reflect the new state of the real world entity. We use Wikipedia revision history %histories 
to distantly supervise a method for automatically learning verbs and state changes. Additionally, our method uses %fact-specific 
constraints
 %such as mutually exclusivisity vs. simultaneous changes of infobox slots
  to effectively map verbs to infobox changes. We observe in our experiments that when state-changing verbs are  added or deleted from an entity's Wikipedia  page text, we can update the entity's infobox with 88\% precision and 76\% recall.
One immediate  application of our verbs is to incorporate as triggers in methods for   updating existing KBs, which are currently mostly static.

\end{abstract}
\section{Introduction}

\begin{figure*}[t]
\begin{center}
\includegraphics[width=16cm,keepaspectratio=true]{figures/motivation.pdf}
\caption{\label{fig:motivation} A snapshot of Kim Kardashian's Wikipedia revision history, highlighting text and infobox changes. In red (and green) are the difference between the page on 05/25/2014 and 05/23/2014: things that are deleted from (and resp. added to) the page.}
\end{center}
\end{figure*}

In recent years there has been a lot of research on extracting relational facts between entities and storing them in knowledge bases (KBs) . These knowledge bases are generally static  \cite{suchanek2007yago,carlson2010toward,fader2011identifying,MitchellCHTBCMG15}
%(which extract facts from Wikipedia infoboxes \cite{suchanek2007yago}) or NELL (which extracts facts from any Web text \cite{carlson2010toward,fader2011identifying}) are generally static.
 They are not updated as the Web changes.
 % when in reality new facts arise while others cease to be valid%or change over time
One approach towards real-time population of KBs is to extract facts from dynamic content of the web such as news \cite{nakashole2012real}. This paper proposes to lean to identify state changes caused by  verbs acting on entities in text. This is different from simply applying the same text extraction pipeline, that created the original KB, to new datasets.
%a \textit{shift} of focus from doing KB updates by extracting facts in text to doing them by 

The benefit of our approach is as follows: (1) In relation extraction, if we consider for example the \textsc{spouse} relation, both \textit{marry} and \textit{divorce} are good patterns for extracting the relation. Therefore, by identifying that they cause different state changes.
%Ndapa: I think most people know this, so we can skip it to keep the text concise
% \textit{marry} signals the start while \textit{divorce} signals the end of the \textsc{spouse} relation; 
 we can update the entity's fact \textit{and} its temporal scope \cite{wijayactp}. (2) Learning state changing verbs
 %  about by verbs 
   can pave ways to learning the ordering of verbs in terms of their  pre- and post-conditions.
   
   % %Ndapa:
  % I think we are being too speculative here. This was fine in a vision paper such as AKBC but here
  % it might seem like over promising. It is therefore better to not ay much here. 
  
   % of state-changing verbs: the entry condition (in terms of KB facts) that must be true for an event expressed by the verb to take place, and the exit condition (in terms of KB facts) that will be true after the event. Such pre- and post-conditions can be useful for (a) learning event sequences %such as scripts \cite{schank2013scripts}, which can be modeled
%as a collection of verbs chained together by pre- and post-condition of their shared entities, (b) for inferring cascading effect of an event via the pre- and post-condition of shared entities in an event sequence, or (c) for inferring unknown states of entities from the verbs they participate in.  

In this paper, we propose to learn state changing verbs using Wikipedia revision history. Our assumption is that when a state-changing event happens to an entity e.g., a marriage, its Wikipedia infobox is updated, by the addition of a new \textsc{spouse} value. The infobox in Wikipedia is  a structured box on the page that contains a set of facts (attribute-value pairs) about the entity. 

At the same time, texts containing verbs that express the event e.g., \textit{wed} may be added to the entity's Wikipedia page.  Figure \ref{fig:motivation} is an example of this happening to an entity's page. Wikipedia revisions over many entities can act as distantly supervision  data for mapping corresponding text and infobox changes. However, these revisions are notoriously  \textit{noisy}.  Many infobox slots can be  updated when a single event happens.
%there is no guarantee that only the infobox slots related to a particular event will be updated. 
For example, when a death happens, slots regarding birth e.g., \textit{birthdate}, \textit{birthplace}, may also be updated. To address these issues, we leverage common sense constraints between state changes e.g., that the start of \textit{deathdate} is mutually exclusive from the \textit{birthdate} or that the start of \textit{birthdate} is simultaneous with the start of \textit{birthplace}.
%to effectively learn infobox changes that relate to a particular event-expressing verb. 

In summary, our contributions are as follows: (1) the preparation and use of  distantly labeled dataset from Wikipedia revision history for learning state changing verbs, and (2) a resource that contains verbs for identifying state changes, which we make available for future research \footnote{URL retracted for blind reviews}.
% learned resource of verbs that is effective for identifying state changes\footnote[1]{We make our dataset and verbs resource available here: http://.../verbs.html}.
\section{Method} \label{sec:method}

\subsection{Data Construction} \label{sec:data}
%Ndapa: histroy is generally singular, 
We construct a dataset from Wikipedia revision history of person entities whose facts change between the year 2007 and 2012 (i.e., have at least one fact in YAGO KB \cite{suchanek2007yago} with a start or end time in this period). We obtain Wikipedia URLs of this set of entities $P$ from YAGO and crawl their revision history%, obtaining any revision their pages have between the year 2007 and 2012
. Given a person $p$, his Wikipedia revision history $R_p$ has a set of ordered dates $T_p$ on which revisions are made to his Wikipedia page (we consider date granularity). Each revision $r_{p, t_p} \in R_p$ is his Wikipedia page at date $t_p$ where $t_p \in T_p$. 

A document $d_{p, t_p}$ in our data set is the \textit{difference}\footnote{a HTML document obtained by ``compare selected revisions"  functionality in Wikipedia} between any two consecutive revisions separated by at least a single date i.e., $d_{p, t_p} = r_{p, t_p+2} - r_{p, t_p}$. Where $r_{p, t_p+2}$ is the \textit{first} revision on date $t_p+2$ and $r_{p, {t_p}}$ is the \textit{last} revision on date $t_p$ (since a page can be revised many times in a day). Our dataset consists of all documents $d_{p, t_p}$, $\forall t_p \in T_p,\ t_p \in [01/01/2007,\ 12/31/2012]$, and $\forall p \in P$; a total of 288,184 documents from revision histories of 16,909 Wikipedia entities.

%For example, Ralph McInerny's Wikipedia page was consecutively revised on the dates of 11/20/2012, 12/26/2012 and 12/29/2012. We find the difference between the last revision to his page on 11/20/2012 and the first revision to his page on 12/29/2012 (since a page can be revised multiple times in a date). This difference\footnote[2]{http://en.wikipedia.org/w/index.php?title=Ralph\_McInerny \&type=revision\&diff=530257160\&oldid=523980632}, a HTML page obtained by ``compare selected revisions"  functionality in Wikipedia, is a document in our dataset. Using this method, we obtain 288,184 documents from revision histories of 16,909 Wikipedia entities. 

Each Wikipedia revision $r_{p, t_p}$ is a set of infobox slots $S_{p, t_p}$ and textual content $C_{p, t_p}$, where each slot $s \in S_{p, t_p}$ is a quadruple, $\langle s_{att}$, $s_{value}$,  $s_{start}$, $s_{end} \rangle$ containing the attribute name (non-empty), the attribute value, and the start and end time for which this attribute-value pair is valid. 

Each document in our dataset is a \textit{difference} between $r_{p, t_p+2}$ and $r_{p, t_p}$, and is a set of infobox changes $\Delta S_{p, t_p}$ and textual changes $\Delta C_{p, t_p}$. Each slot change $\delta s \in \Delta S_{p, t_p} = \langle s_{att}$, $\delta s_{value}$,  $\delta s_{start}$, $\delta s_{end} \rangle$, where $\delta s_{value}$,  $\delta s_{start}$, or $\delta s_{end}$, whenever not empty, is prefixed with $+$ or $-$ to indicate whether they are added or deleted in $r_{p, t_p+2}$. Similarly, each text change $\delta c \in \Delta C_{p, t_p}$ is prefixed with $+$ or $-$ to indicate whether they are added or deleted in $r_{p, t_p+2}$. For example, n Figure \ref{fig:motivation}, a document $d_{kim,\ 05/23/2014} = r_{kim, 05/25/2014} - r_{kim, 05/23/2014}$ is a set of slot changes: $\langle\textsc{spouse}$, \textbf{$+$}\footnotesize ``Kanye West"\normalsize,  $+$\footnotesize ``2014"\normalsize, \footnotesize`` "\normalsize$\rangle$, $\langle\textsc{partner}$, $-$\footnotesize``Kanye West"\normalsize,  $-$\footnotesize``2012-present; engaged"\normalsize, \footnotesize`` "\normalsize$\rangle$ and a set of text changes: $+$\footnotesize``Kardashian and West were married in May 2014"\normalsize, $-$\footnotesize``She began dating West"\normalsize, $-$\footnotesize``they became engaged in October 2013"\normalsize.

For each $d_{p, t_p}$, we use $\Delta S_{p, t_p}$ to label the document and $\Delta C_{p, t_p}$ to extract features for the document. We label $d_{p, t_p}$ that has $\langle s_{att}, +\delta s_{value}, *, *\rangle \in \Delta S_{p, t_p}$ or $\langle s_{att}, *, +\delta s_{start}, *\rangle$ $\in \Delta S_{p, t_p}$ with the label \textit{begin-}$s_{att}$ and $d_{p, t_p}$ that has $\langle s_{att}, *, *, +\delta s_{end}\rangle$ $\in \Delta S_{p, t_p}$ with the label \textit{end-}$s_{att}$. The label represents the state change that happens in $d_{p, t_p}$. For example, in Figure \ref{fig:motivation}, $d_{kim,\ 05/23/2014}$ is labeled with \textit{begin-spouse} and \textit{end-partner}. As features, for each labeled $d_{p,t_p}$, we extract verbs (or verbs+prepositions) $v \in \Delta C_{p, t_p}$ that have ($v_{subject}$, $v_{object}$) = (\scriptsize$arg1$\normalsize, \scriptsize$arg2$\normalsize) or ($v_{subject}$, $v_{object}$) = (\scriptsize$arg2$\normalsize,  \scriptsize$arg1$\normalsize), where \scriptsize$arg1$\normalsize = $p$ and $\langle s_{att}, $\scriptsize$arg2$\normalsize$, *, *\rangle$ or $\langle s_{att}, *, $\scriptsize$arg2$\normalsize$, *\rangle$ or $\langle s_{att}, *, *, $\scriptsize$arg2$\normalsize$\rangle$ is $\in \Delta S_{p, t_p}$.

%  is the document's entity $p$ and whose object (\textit{arg2}) is any of its $\delta s$ value (or vice versa). We use 90\% of our labeled documents as training and test on the remaining 10\%. The task is to predict for each document, the label of the document given its verbs features. 

%We define an infobox attribute of an entity e.g., \textsc{spouse} to \textit{begin} when a new value or a begin time is added to the attribute slot and to \textit{end} when an end time is being added to the slot. Using regular expression to detect whether a new value, a start, or an end time is being added to infobox slots of a document, we automatically label each document with ``begin-\{attribute\_name\}" or ``end-\{attribute\_name\}". So a document that contains an addition of a new value in the \textsc{spouse} slot will be labeled ``begin-spouse", while a document that contains an addition of end time in the \textsc{spouse} slot will be labeled ``end-spouse". 

\subsection{Model}
We use a Maximum Entropy (\textsc{MaxEnt}) classifier given the set of training data = \{(\textbf{v}$_{d_{\ell}}$, y)\} where \textbf{v}$_{d_{\ell}} =$ ($v_1$, $v_2$, ... $v_{|V|}$) $\in R^{|V|}$ is the $|V|$-dimensional representation of a labeled document $d_{\ell}$ where $V$ is the set of all verbs in our training data, and $y$ is the label of $d_{\ell}$ as defined in \ref{sec:data}.

These training documents are used to estimate a set of weight vectors \textbf{w} = \{\textbf{w}$_1$, \textbf{w}$_2$, ... \textbf{w}$_{|Y|}$\}, one for each label $y \in Y$, the set of all labels in our training data. The classifier can then be applied to classify an unlabeled document $d_{\textit{u}}$ using: 
\scriptsize
 \begin{equation}
	p(y|\textbf{v}_{d_{\textit{u}}}) = \frac{\mathrm{exp}  (\textbf{w}_{y} \cdot \textbf{v}_{d_{\textit{u}}})}{\sum_{y'} \mathrm{exp} (\textbf{w}_{y'} \cdot \textbf{v}_{d_{\textit{u}}})} \label{eqn:maxent}
\end{equation}
\normalsize
\subsection{Feature Selection using Constraints}

While feature weights obtained by \textsc{MaxEnt} allow us to identify verbs that are good features for predicting a particular state change label, our distantly supervised training data is inherently noisy. %There is no guarantee that at any revision, the changed slots are related to the event being expressed in text. 
For example, when death happens, birth-related information in the infobox may also be updated. This can lead to incorrect state change prediction. While using many distantly labeled data may help%reduce the effect of noise
, improvements may be possible by leveraging constraints among state changes to select consistent verb features for each change. 

The constraints that we use can be categorized into: (1) mutual exclusion (\textit{Mutex}) which indicate that mutex state changes should not \textit{typically} happen at the same time e.g., update on \textit{birthdate} should not \textit{typically} happen with update on \textit{deathcause}. %or that the start of \textit{spouse} is mutually exclusive with the end of \textit{spouse}. 
A good \textit{base} verb\footnote[3]{The verb root or base form of a verb} for one change e.g., ``marry" for \textit{begin-spouse} is therefore not a good feature for \textit{end-spouse} (mutex with \textit{begin-spouse}). (2) Simultaneous (\textit{Sim}) constraints which indicate that simultaneous state changes should \textit{typically} happen at the same time e.g., update on \textit{birthdate} should \textit{typically} happen with other birth-related updates. %on \textit{birthplace}, \textit{birthname}, \textit{birthcity}, etc. 
A good base verb for one state change e.g., ``die" for \textit{begin-deathdate} is therefore a good feature for \textit{begin-deathdate}'s simultaneous changes: \textit{begin-deathplace}, \textit{begin-deathcause}, etc. We obtain such constraints using heuristics on our label names.%, labels that share the same prefix such as \textit{begin-birth*} have simultaneous constraints while labels that share the same suffix but different prefix such as \textit{begin-spouse} and \textit{end-spouse} or \textit{begin-yearsactive} and \textit{end-yearsactive} have mutually exclusive constraints.

Given a set of constraints, a set of labels $Y$, and a set of base verbs $B$ in our training data, we solves a Mixed-Integer Program (MIP) for each base verb $b \in B$ to estimate whether $b$ should be a feature for state change $y \in Y$. 

We obtain label membership probabilities \{$P(y | b) = count(y, b) / \sum_{y'} count(y', b) $\} from our training data. The MIP takes the scores $P(y | b)$ and constraints as input and produces a bit vector of labels \textbf{a}$_{b}$ as output, each bit  $a_{b}^{y} \in \{0,\ 1\}$ representing whether $b$ should be a feature for $y$/not.

The MIP formulation for a base verb $b$ is presented in Equation \ref{eqn:mip}. For each $b$, this method tries to maximize the sum of scores of selected labels, after penalizing for violation of label constraints. Let $\zeta_{y, y'}$ be slack variables for \textit{Sim} constraints, and $\xi_{y, y'}$ be slack variables for \textit{Mutex} constraints. 

\scriptsize
\begin{equation}
\begin{aligned}
& \operatorname*{maximize}_{\textbf{a}_b,\ \zeta_{y, y'},\ \xi_{y, y'}}
& & \bigg( \sum_{y} a_{b}^{y}\ *\ P(y | b)\ - \sum_{\langle y, y'\rangle \in Sim} \zeta_{y, y'}\ - \sum_{\langle y, y'\rangle \in Mutex} \xi_{y, y'}\bigg) \\ 
& \text{subject to} & & \big(a_{b}^{y} - a_{b}^{y'}\big)^2 \leq \zeta_{y, y'},\ \ \ \forall \langle y, y'\rangle \in Sim \\
& & & a_{b}^{y} + a_{b}^{y'} \leq 1 + \xi_{y, y'},\ \ \ \forall \langle y, y'\rangle \in Mutex \\
& & & \zeta_{y, y'},\ \xi_{y, y'} \geq 0,\ a_{b}^{y} \in \{0, 1\},\ \ \ \forall y, y'
\label{eqn:mip}
\end{aligned}
\end{equation}
\normalsize

Solving MIP per base verb is fast. To make it even more efficient, we reduce the number of labels considered per base verb i.e., we only consider a label $y$ to be a candidate for $b$ if $\exists\ v_i \in V$ s.t. $w_y^i > 0$ and $b$ = base form of $v_i$ (after removing preposition). Then, we need to only solve MIP for base verbs that have non-empty candidate labels. 

After we output \textbf{a}$_b$ for each base verb, we do feature selection on the learned verb features of each label. We only select a verb $v_i$ to be a feature for $y$ if the learned weight $w_y^i > 0$ and $a_b^{y} = 1$, where $b$ = the base form of $v_i$. Essentially for each label, we are choosing verb features that have positive weights and are consistent for the label.
\section{Experiments}

<<<<<<< HEAD
We use 90\% of our labeled documents as train and test on the remaining 10\%. Since our data is noisy, we manually go through our test data to discard documents that have incorrect label to the change in its text. The task is to predict for each document, the label of the document given its verbs features. We compute precision, recall, and F1 values of our predictions of the test set and compare the values before and after feature selection (Fig. \ref{fig:result}).
=======
We use 90\% of our labeled documents as train and test on the remaining 10\%. Since revision history data is noisy, we manually go through our test data to discard documents that have incorrect infobox  labels by looking the  text that  changed. The task is to predict for each document (revision), the label (infobox slot value) of the document given its verbs features. We compute precision, recall, and F1 values of our predictions  and compare the values before and after feature selection (Figure \ref{fig:result}).
>>>>>>> upstream/master

%\begin{table}
%\begin{small}
%\begin{center}
%\begin{tabular}{|l|r|r|r|}
%\hline
%Method & Precision & Recall & F1 \\
%\hline
%\textsc{MaxEnt} & 0.82 & 0.79 & 0.80 \\
%\hline
%\textsc{MaxEnt} + MIP & 0.88 & 0.76 & 0.82 \\
%\hline
%\end{tabular}
%\caption{\label{table:performance} Results of predicting state change label using verb features.}
%\end{center}
%\end{small}
%\end{table}

<<<<<<< HEAD
We observe the value of doing feature selection by asserting constraints in an MIP formulation% in Figure \ref{fig:result}
. Feature selection improves precision without reducing too much recall; thus resulting in a better F1 (Fig. \ref{fig:result}). Some inconsistent verb features for the labels were removed by asserting constraints. For example, before feature selection, the verbs: ``marry" and ``be married to" were high-weighted features for both \textit{begin-spouse} and \textit{end-spouse}. After asserting constraints that \textit{begin-spouse} is mutex with \textit{end-spouse}, these verbs (whose base form is ``marry") are filtered out from the features of \textit{end-spouse}. We show some of the learned verb features (after feature selection) for some state change labels in (Table \ref{table:verbs}).
=======
We observe the value of doing feature selection by asserting constraints in an MIP formulation in Figure \ref{fig:result}. Feature selection improves precision without significantly reducing  recall; resulting in a better F1. By asserting constraints, some of the  inconsistent verb features for the labels were removed. For example, before feature selection, the verbs: ``marry", ``marry in" and ``be married to" were high-weighted features for both \textit{begin-spouse} and \textit{end-spouse}. After asserting constraints that \textit{begin-spouse} is mutex with \textit{end-spouse}, these verbs (whose base form is ``marry") are filtered out from the features of \textit{end-spouse}. We show some of the learned verb features (after feature selection) for some of the labels in (Table \ref{table:verbs}).
>>>>>>> upstream/master

\begin{figure}
\begin{center}
\includegraphics[width=5cm,keepaspectratio=true]{figures/result.pdf}
\caption{\label{fig:result} Results of predicting state change labels (infobox types) using verb features.}
\end{center}
\end{figure}

\begin{table}[t]
\begin{scriptsize}
\begin{center}
\begin{tabular}{|l|l|}
\hline
Label & Verb \\
\hline
\textit{begin-} &+(arg1) die on (arg2), +(arg1) die (arg2), \\
\textit{deathdate}& +(arg1) pass on (arg2) \\
\hline
%\textit{begin-deathplace} &+(arg1) die in (arg2), +(arg1) die at (arg2), +(arg1) move to (arg2) \\
%\hline
\textit{begin-} &+(arg1) be born in (arg2), +(arg1) bear in (arg2), \\
\textit{birthplace}&+(arg1) be born at (arg2) \\
\hline
\textit{begin-} &+(arg1) succeed (arg2), +(arg1) replace (arg2), \\
\textit{predecessor}& +(arg1) join cabinet as (arg2), +(arg1) join as (arg2) \\
\hline
\textit{begin-} &+(arg1) lose \textbf{seat} to (arg2), +(arg1) resign on (arg2), \\
\textit{successor}& +(arg1) resign from post on (arg2) \\ %+(arg1) lose election to (arg2) \\
\hline
%\textit{begin-} &+(arg1) work as (arg2), +(arg1) nominate for (arg2), \\
%\textit{occupation}& +(arg1) establish as (arg2) \\
%\hline
\textit{begin-} &+(arg1) be appointed on (arg2), +(arg1) serve from (arg2), \\
\textit{termstart}& +(arg1) be elected on (arg2) \\
\hline
%\textit{begin-termend} &+(arg1) resign on (arg2), +(arg1) step down in (arg2), \\
%& +(arg1) flee in (arg2) \\
%\hline
%\textit{begin-office} &+(arg1) be appointed as (arg2), \\
%& +(arg1) serve as (arg2), +(arg1) be appointed (arg2) \\
%\hline
\textit{begin-} &+(arg1) marry on (arg2), +(arg1) marry (arg2), \\
\textit{spouse}& +(arg1) be married on (arg2), -(arg1) be engaged to (arg2) \\
\hline
\textit{end-spouse} &+(arg1) file \textbf{for divorce} in (arg2), +(arg1) die on (arg2), \\
& +(arg1) divorce in (arg2) \\%+(arg1) announce \textbf{separation} on (arg2) \\
\hline
%\textit{begin-children} &+(arg1) have \textbf{child} (arg2), +(arg1) raise daughter (arg2), \\
%& +(arg1) raise (arg2) \\
%\hline
%\textit{begin-party} &+(arg1) launch (arg2), +(arg1) be elected as (arg2), +(arg1) be elected (arg2) \\
%\hline
%\textit{begin-} &+(arg1) graduate from (arg2), +(arg1) attend (arg2), \\
%\textit{almamater}& +(arg1) be educated at (arg2) \\
%\hline
%\textit{begin-awards} &+(arg1) be awarded (arg2), +(arg1) be named on (arg2), \\
%& +(arg1) receive (arg2) \\
%\hline
\textit{begin-} &+(arg1) start career with (arg2), \\
\textit{youthclubs}& +(arg1) begin \textbf{career} with (arg2), +(arg1) start with (arg2) \\
\hline
%\textit{begin-clubs} &+(arg1) play for (arg2), +(arg1) play during career with (arg2),\\
%& +(arg1) sign with (arg2), +(arg1) complete \textbf{move} to (arg2) \\
%\hline
%\textit{begin-} &+(arg1) make \textbf{appearance} for (arg2), \\
%\textit{nationalteam}& +(arg1) make debut for (arg2), +(arg1) play for (arg2) \\
%\hline
\end{tabular}
\caption{\label{table:verbs} Comparison of verb phrases learned before and after feature selection for various labels (infobox types).}
% The texts in bold are (prep+) noun that occur most frequently with the combination of the (verb, label) in the train data.}
\end{center}
\end{scriptsize}
\end{table}
\normalsize

\section{Related Work}

\textbf{Learning from Wikipedia Edit History.}
Wikipedia edit history has been exploited in a number
of language understanding problems.
However, prior methods were targeted for various tasks different form ours.
  A popular task in this regard is that of
Wikipedia edit history categorization\cite{daxenberger2013automatically}.  This task
involves characterizing  a given change instance as one of many possible categories such as
such as spelling error correction, paraphrase or vandalism to edits in a document. 
\cite{DaxenbergerG12}  came up with a 21 category edit classification
taxonomy.  Other tasks to leverage Wikipedia edit history include:
 spelling error correction, summarization, preposition error
 correction, sentence compression, bias detection, and
 textual  entailment \cite{Nelken08miningwikipedia,Cahill13robustsystems,Zanzotto_expandingtextual,RecasensDJ13}.
These studies are concerned with coarse grained change type classification as opposed
to the establishing a specific correspondence between text changes, at a word level,  to  infobox changes.

\textbf{Learning State Changing Verbs.}
Very few works have studied the problem of detecting state changing verbs.
\cite{HosseiniHEK14} learned state changing verbs in the context of solving arithmetic word problems.
The learned the effect of the words such as add, subtract on the the current state. 
The VerbOcean resource was automatically generated from the Web\cite{Chklovski04}. The authors  studied the problem of fine-grained semantic relationships between verbs from Web document. They learn relations such as  if someone has bought an item, they may sell it at a later time. This then involves capturing empirical regularities  if ``X buys Y'' happens before ``X
sells Y'', for the same X and Y values in a given context. Unlike the work we present here, the methods of \cite{Chklovski04,HosseiniHEK14}  do not make a connection to knowledge base relations such as Wikipedia infoboxes.
In a vision paper, \cite{Wijaya2014akbc} give high level descriptions of  a number of possible methods for learning state changing methods but did not implement any of them.

\section{Conclusion}


\section*{Acknowledgments}
We thank members of the NELL team at CMU for their helpful comments.
This research was supported by
DARPA under contract number FA8750-13-2-0005.

% include your own bib file like this:
\bibliographystyle{acl}
\bibliography{precondition}


\end{document}
